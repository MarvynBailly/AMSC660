\documentclass[12pt]{report}
\usepackage{commands}

\begin{document}

\large
\begin{center}
AMSC 660 Homework 2\\
Due 09/13/23\\
By Marvyn Bailly\\
\end{center}
\normalsize
\hrule


%Good luck my man
%---------------%
%---Problem 1---%
%---------------%




\begin{problem}%[vskip]
\subsection*{Problem 1}

Consider the polynomial space $\mathcal{P}_n(x),x\in[-1,1]$. Let $T_k, k =0,1,\dots,n,$ be the Chebyshev basis in it. The Chebyshev polynomials are defined via
\[
    T_k = \cos(k \arccos(x)).
\]
\begin{enumerate}
    \item [(a)] Use the trigonometric formula
    \[
        \cos(a) + \cos(b) = 2 \cos \paren{\frac{a+b}{2}}\cos\paren{\frac{a-b}{2}}
    \]
    to derive the three-term recurrence relationship for the Chebyshev polynomials
    \[
        T_0(x) = 1, ~~~ T_1(x) = x, ~~~ T_{k+1} = 2x T_k(x) - T_{k-1}(x), ~~ k=1,2,\dots.
    \]


    \item [(b)] Consider the differentiation map
    \[
        \dd{}{x}: \mathcal{P}_n \to \mathcal{P}_{n-1}.
    \]
    Write the matrix of the differentiation map with respect to the Chebyshev bases in $\mathcal{P}_n \and \mathcal{P}_{n-1}$ for $n = 7$. Hint: you might find helpful properties of Chebyshev polynomials presented in Section 3.3.1 of Gil, Segure, Temme, ”Numerical
    Methods For Special Functions”. Chapter 3 of this book is added to Files/Refs
    on ELMS.


\end{enumerate}


\subsection*{Solution}
\begin{proof}

Consider the polynomial space $\mathcal{P}_n{x},x\in[-1,1]$. Let $T_k, k =0,1,\dots,n,$ be the Chebyshev basis in it. The Chebyshev polynomials are defined via
\[
    T_k = \cos(k \theta),
\]
where $\theta = \arccos(x)$.

\begin{enumerate}
    \item [(a)]
    Applying the trigonometric formula
    \[
        \cos(a) + \cos(b) = 2 \cos \paren{\frac{a+b}{2}}\cos\paren{\frac{a-b}{2}},
    \]
    gives
    \begin{align*}
        T_{k-1}(x) + T_{k+1}(x) &= \cos((k-1)\theta) + \cos((k+1)\theta)\\
        &= 2 \cos \paren{ \frac{(k-1)\theta + (k+1)\theta}{2}}\cos\paren{\frac{(k-1)\theta - (k+1)\theta}{2}}\\
        &= 2 \cos (k\theta) \cos(-\theta)\\
        &= 2 \cos(\theta) \cos(k\theta)\\
        &= 2 x T_k(x).
    \end{align*}
    Then rearranging the terms gives the three-term recurrence relationship to be
    \[
        T_{k-1}(x) + T_{k+1} = 2x T_k x \implies T_{k+1} = 2xT_k(x) - T_{k-1}(x)
    \]

    \item [(b)]
    Consider the differentiation map
    \[
        \dd{}{x}: \mathcal{P}_n \to \mathcal{P}_{n-1}.
    \]
    We wish to write the matrix of the differentiation map with respect to the Chebyshev bases in $\mathcal{P}_n \and \mathcal{P}_{n-1}$ for $n = 7$.

    From the given resources we have the following relation for the Chebyshev polynomial derivatives
    \[
        \begin{cases}
            T_0(x) = T_1'(x),\\
            T_1(x) = \frac{1}{4} T_2'(x),\\
            T_n(x) = \frac{1}{2}\paren{\frac{T_{n+1}'}{n+1} - \frac{T_{n-1}'}{n-1}}.
        \end{cases}
    \]
    Notice that we can rearrange the terms to find
    \begin{align*}
        T_n &= \frac{1}{2}\paren{\frac{T_{n+1}'}{n+1} - \frac{T_{n-1}'}{n-1}}\\
        2T_n &= \frac{T_{n+1}'}{n+1} - \frac{T_{n-1}'}{n-1}\\
        \frac{T_{n+1}'}{n+1} &= 2T_n + \frac{T_{n-1}'}{n-1}\\
        T_{n+1}' &= (n+1)\paren{2T_n + \frac{1}{n-1}T_{n-1}'}\\
        T_{n+1}' &= 2(n+1)T_n + \frac{n+1}{n-1}T_{n-1}'\\
        T_{n}' &= 2nT_{n-1} + \frac{n}{n-2}T_{n-2}'.
    \end{align*}
    We can use this relation to compute the first seven derivatives
    \begin{align*}
        T_0' &= 0,\\
        T_1' &= T_0,\\
        T_2' &= 4T_1,\\
        T_3' &= 6T_2 + 3T_1' = 6T_2 + 3T_0,\\
        T_4' &= 8T_3 + \frac{4}{2}T_2' = 8T_3 + 8T_1,\\
        T_5' &= 10T_4 + \frac{5}{3}T_3' = 10T_4 + 10T_2 + 5T_0,\\
        T_6' &= 12T_5 + \frac{6}{4}T_4' = 12T_5 + 12T_3 + 12T_1,\\
        T_7' &= 14T_6 + \frac{7}{5}T_5' = 14 T_6 + 14 T_4 + 14 T_2 + 7T_0.
    \end{align*}
    Thus the matrix of the differentiation map is given by
    \[
        A = \begin{pmatrix}
            0 & 1 & 0 & 3 & 0 & 5  & 0  & 7\\
            0 & 0 & 4 & 0 & 8 & 0  & 12 & 0\\
            0 & 0 & 0 & 6 & 0 & 10 & 0 & 14\\
            0 & 0 & 0 & 0 & 8 & 0 & 12 & 0\\
            0 & 0 & 0 & 0 & 0 & 10 & 0 & 14\\
            0 & 0 & 0 & 0 & 0 & 0 & 12 & 0\\
            0 & 0 & 0 & 0 & 0 & 0 & 0 & 14\\
        \end{pmatrix},
    \]
    where the columns of $A$ correspond to the Chebyshev bases for $T_i'$ for $0\leq i \leq 7$, i.e. the $A_{00}$ entry corresponds to the $T_0$ element of $T_0'$.


\end{enumerate}

\end{proof}
\end{problem}




%---------------%
%---Problem 2---%
%---------------%


\begin{problem}%[vskip]
\subsection*{Problem 2}

Let $A = (a_{ij})$ be an $m \times n$ matrix.

\begin{enumerate}
    \item [(a)] Prove that the $l_1$-norm of $A$ is
    \[
        \| A \|_1 = \max_j \sum_i |a_{ij}|,
    \]
    i.e., the maximal column sum of absolute values. Find the maximizing vector.
    \item [(b)] Prove that the max-norm or $l_\infty$-norm of $A$
    \[
        \| A \|_{\max} = \max_i \sum_j |a_{ij}|,
    \]
    i.e., the maximal row sum of absolute values. Find the maximizing vector.
\end{enumerate}

\subsection*{Solution}
\begin{proof}

Let $A = (a_{ij}) = (a_1 | a_2 | \cdots | a_n)$ be an $m \times n$ matrix. 
\begin{enumerate}
    \item [(a)]
    Let $k$ be the index such that $\max_{j} \| a_j \|_1 = \|a_{k}\|_1$. By definition we have
    \begin{align*}
        \|A\|_1 &=  \max_{\|v\|_1 = 1}\| Av \|_1\\
                &=  \max_{\|v\|_1 = 1}\left\| \sum_{j=1}^n a_{j}v_j \right\|_1\\
                &\leq \max_{\|v\|_1 = 1}\sum_j^n \|a_j v_j \|_1\\ 
                &=\max_{\|v\|_1 = 1} \sum_j^n |v_j| \| a_j\|_1\\
                &\leq \max_{\|v\|_1 = 1} \paren{\sum_j^n |v_j|} \|a_{k}\|_1\\
                &= \| a_k \|_1\\
                &= \max_j \| a_j \|_1\\
                &= \max_j \sum_i |a_{ij}|.
    \end{align*}
    Notice that if we let $v = e_k$, we achieve equality  
    \[
        \|A\|_1 = \|A e_k\|_1 = \|a_k\|_1 = \max_{j}\sum_i |a_{ij}|.
    \]


    \item [(b)]
    By definition we have
    \begin{align*}
        \| A \|_\infty &= \max_{\|v\|_\infty = 1 }\| Av \|_\infty\\
        &= \max_{\|v\|_\infty = 1 }\left\| \sum_j^n a_jv_j   \right\|_\infty\\
        &\leq \max_{\|v\|_\infty = 1}  \left\| \sum_j^n \|v\|_\infty a_j  \right\|_\infty\\
        &= \left\| \sum_j^n a_j \right\|_\infty\\
        &= \max_i \sum_j^n |a_{ij}|.
    \end{align*}
    Notice that if we define the vector $v$ such that $v_i = \text{Sign}(a_{ij})$, then $a_{ij}v_i = |a_{ij}|$ which means
    \[
        \| A \|_\infty = \|Av\|_\infty = \left\| \sum_j^n |a_j| \right\|_\infty = \max_i \sum_j^n|a_{ij}|
    \] 


\end{enumerate}

\end{proof}
\end{problem}




%---------------%
%---Problem 3---%
%---------------%


\begin{problem}%[vskip]
\subsection*{Problem 3}

Consider the matrix
\[
    A = \begin{pmatrix}
        1 & 10\\
        0 & 1
    \end{pmatrix}.
\]
\begin{enumerate}
    \item [(a)] Find the Jordan form of $A$.
    \item [(b)] Find the $2$-norm of $A$.
\end{enumerate}

\subsection*{Solution}
\begin{proof}

Consider the matrix
\[
    A = \begin{pmatrix}
        1 & 10\\
        0 & 1
    \end{pmatrix}.
\]

\begin{enumerate}
    \item [(a)]
    Thus $\lambda = 1$ with algebraic multiplicity $2$ are the eigenvalues of $A$. Next, let's find eigenspace corresponding to $\lambda = 1$ by solving $Av = \lambda v$, where $v$ is the corresponding eigenvector
    \begin{align*}
        \begin{pmatrix}
            1 - \lambda & 10\\
            0 & 1 - \lambda
        \end{pmatrix}\begin{pmatrix}
            a_1 \\ a_2
        \end{pmatrix} &= 0\\
        \begin{pmatrix}
            0 & 10\\
            0 & 0
        \end{pmatrix}\begin{pmatrix}
            a_1 \\ a_2 
        \end{pmatrix}   &= 0,      
    \end{align*}
    solving the system gives
    \[
        \begin{pmatrix}
            a_1 \\ a_2
        \end{pmatrix} = \begin{pmatrix}
            a_1 \\ 0
        \end{pmatrix},
    \]
    and letting $a_1 = 1$ we get the eigenvector $v_1 = (1, 0)^T$. Since $\lambda$ has algebraic multiplicity $2$, we can find the second eigenvector in the corresponding eigenspace using the generalized eigenvector of the form
    \begin{align*}
        (A - I\lambda)v_2 = v_1 \implies \begin{pmatrix}
            0 & 10\\
            0 & 0
        \end{pmatrix} \begin{pmatrix}
            b_1 \\ b_2
        \end{pmatrix} = \begin{pmatrix}
            1 \\ 0
        \end{pmatrix} \implies \begin{pmatrix}
            10 b_2 \\ 0
        \end{pmatrix} = \begin{pmatrix}
            1 \\ 0
        \end{pmatrix},
    \end{align*}
    thus we have that $v_2 = (0, 1/10)^T$. Then to find the Jordan form we compute
    \[
        P^{-1}AP = \begin{pmatrix}
            1 & 0 \\ 0 & 1/10
        \end{pmatrix}^{-1}\begin{pmatrix}
            1 & 10\\ 0 & 1
        \end{pmatrix}\begin{pmatrix}
            1 & 0 \\ 0 & 1/10
        \end{pmatrix} = \begin{pmatrix}
            1 & 1\\ 0 & 1
        \end{pmatrix} = J.
    \]

    \item [(b)]
    Recall that
    \[
        \|A\|_2 = \sqrt{\lambda_\text{max}(A^*A)} = \sigma_\text{max}.
    \]
    Notice that
    \[
        A^* A = \begin{pmatrix}
            1 & 0\\
            10 & 1 
        \end{pmatrix}\begin{pmatrix}
            1 & 10\\
            0 & 1
        \end{pmatrix} = \begin{pmatrix}
            1 & 10\\
            10 & 101
        \end{pmatrix},
    \]
    whose characteristic polynomial is
    \[
        (1 - \lambda)(101 - \lambda) - 100 = \lambda^2 -100\lambda + 1 = 0.
    \]
    Solving using the quadratic formula yields
    \[
        \lambda_{1,2} = 51 \pm 10\sqrt{26}.
    \]
    Thus 
    \[
        \|A\|_2 = \sqrt{51 \pm 10\sqrt{26}} \approx 10.09
    \]

\end{enumerate}

\end{proof}
\end{problem}






\end{document}