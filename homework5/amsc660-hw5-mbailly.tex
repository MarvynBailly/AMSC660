\documentclass[12pt]{report}
\usepackage{commands}

\begin{document}

\large
\begin{center}
AMSC 660 Homework 5\\
Due Next Week\\
By Marvyn Bailly\\
\end{center}
\normalsize
\hrule

%Good luck my man
%---------------%
%---Problem 1---%
%---------------%

\def\L{{\mathcal{L}}}



\begin{problem}%[vskip]
\subsection*{Problem 1}


\begin{enumerate}
    \item [(a)] Consider the set $\L$ of all $n \times n$ lower-triangular matrices with positive diagonal entries.
    \begin{itemize}
        \item Prove that the product of any two matrices in $\L$ is also in $\L$.
        \item Prove that the inverse of any matrix in $\L$ is also is $\L$.
    \end{itemize}

    \item [(b)] Prove that the Cholesky decomposition for any $n \times n$ symmetric positive definite matrix is unique. \textit{Hit. Proceed from converse. Assume that there are two Cholesky decomposition $A = LL^T$ and $A = MM^T$. Show that then $M^{-1}LL^TM^{-T} =  I$. Conclude that $M^{-1}L$ must be orthogonal. Then use item (a) of this problem to complete the argument.}
\end{enumerate}

\subsection*{Solution}
\begin{proof}

\begin{enumerate}
    \item [(a)]
    Consider the set $\L$ of all $n \times n$ lower-triangular matrices with positive diagonal entries. Let $A,B \in \L$, then if $C = AB$, the elements of $C$ are given by
    \[
         c_{ij} = \sum_{k=1}^n a_{ik} b_{kj}.
    \]
    Thus for $c_{ij}$ with $i=j$
    \begin{align*}
        c_{ii} &= a_{i1}b_{1i} + a_{i2}b_{2i} +  \cdots +a_{ii-1}b_{i-1i} + a_{ii}b_{ii} + a_{ii+1}b_{i+1i} + \cdots + a_{in-1}b_{n-1j} + a_{in}b_{nj}\\
        &= a_{i1}0 + a_{i2}0 +  \cdots +a_{ii-1}0 + a_{ii}b_{ii} + 0b_{i+1i} + \cdots + 0b_{n-1j} + 0b_{nj}\\
        &= a_{ii}b_{ii}.
    \end{align*}
    So $c_{ii} = a_{ii}b_{ii} > 0$. If $i < j$, $c_{ij}$ is given by
    \begin{align*}
        c_{ij} &= a_{i1}b_{1j} +  \cdots + a_{ii}b_{ij} +  \cdots + a_{ij}b_{jj} + \cdots + a_{in}b_{nj}\\
        &= a_{i1}0 + \cdots + a_{ii}0 + \cdots + 0 b_{jj} + \cdots + 0b_{nj}\\
        &= 0.
    \end{align*}
    So $c_{ij} = 0$ for $i < j$. Therefore $C$ is lower triangular with positive diagonal elements and thus $C \in \L$. Next, let $A \in \L$. Let $B = A^{-1}$ where $b_{ij}$ are the elements of $B$. Then $AB = C = I$. Notice that, $c_{ii} = a_{ii}b_{ii} = 1$ and since $a_{ii} > 0$, then $b_{ii} >0$. To see that the inverse is also lower triangular, denote the columns of $B$ as $b_i$ for $1 \leq i \leq n$. Then
    \[
         AB = A[b_1 | \cdots | b_n] = [Ab_1 | \cdots | Ab_n] = I,
    \]
    and thus
    \[
         Ab_i = e_i,
    \]
    where $e_i$ has $1$ in the ith position and zeros elsewhere for $1 \leq i \leq n$. Then since $e_i$ has zeros above the $i$th row and $A$ is lower triangular, $b_i$ has only zeros above the $i$th row. Thus $B$ is lower triangular. Since $A^{-1}$ is lower triangular and has positive elements along the main diagonal we have shown that $A \in \L \implies A^{-1} \in \L$.



    \item [(b)] Consider $A$ to be an $n \times n$ symmetric positive definite matrix which has two Cholesky decomposition $A = LL^T$ and $A = MM^T$. Observe that
    \begin{align}
        &LL^T = MM^T \notag\\ 
        \iff &M^{-1}LL^T = M^T \notag\\
        \iff &M^{-1}LL^TM^{-T} =I \notag\\
        \iff &(M^{-1}L)(M^{-1}L)^{-T} = I . 
    \end{align}
    From (a), we know that the product and inverse of lower triangular matrices with positive diagonal entries are lower triangular with positive diagonal entries, and thus $M^{-1}L$ is lower triangular with positive diagonal entries. Rearranging the terms of Eq. (1) gives
    \[
        (M^{-1}L) = (M^{-1}L)^{T},
    \]
     and since both sides of the equation are lower triangular, we conclude that $M^{-1}L$ must be diagonal. Let's say that the diagonal elements of $M^{-1}L$ are given by $d_1,\dot,d_n$ and so the diagonal elements of $(M^{-1}L)^{-T}$ are given by $d_1^{-1}, \dots, d_n^{-1}$. By Eq. (1), $d_i\cdot d_i^{-1} = 1$ which shows that $d_i = 1,  \forall i = 1, \dots n$. Therefore
     \[
         M^{-1}L = I \implies L = M,
     \]
    which shows that the Cholesky decomposition of a SPD matrix is unique.

\end{enumerate}

\end{proof}
\end{problem}


%---------------%
%---Problem 2---%
%---------------%


\begin{problem}%[vskip]
\subsection*{Problem 2}

The Cholesky algorithm is the cheapest way to check if a symmetric matrix is positive definite.

\begin{enumerate}
    \item [(a)] Program the Cholesky algorithm. If any $L_{jj}$ turns out to be either complex or zero, make it terminate with a message: "The matrix is not positive definite."
    \item [(b)] Generate a symmetric $100 \times 100$ as follows: generate $\tilde{A}$ with entries being random numbers uniformly disturbed in $(0,1)$ and defined $A := \tilde{A} + \tilde{A}^T$. Use the Cholesky algorithm to check if $A$ is a symmetric positive definite. Compute the eigenvalue of $A$ using a standard command (\verb+eig+), find minimal eigenvalue, and check if the conclusion of your Cholesky-based test for positive definiteness is correct. If $A$ is positive definite, compute its Cholesky factorization using a stand command and print the norm of the difference of the Cholesky factors computed by your routine and by standard one.
    \item [(c)] Repeat item (b) with $A$ defined by $A = \tilde{A}^T\tilde{A}$. The point of this task is to check that your Cholesky routine works correctly. 
\end{enumerate}

\subsection*{Solution}
\begin{proof}

\begin{enumerate}
    \item [(a)]
    I coded the check as
    \begin{python}
    if(L(j,j) == 0 || ~isreal(L(j,j)))
        fprintf("The matrix is not positive definite")
        result = 0;
        return 
    end
    \end{python}

    \item [(b)]
    I made the following code snippets to perform part (b). Here is the Cholesky method with the check from part (a).
    \begin{python}
    function result = cholesky(A)
        n = size(A);
        L = zeros(n,n)a;
    
        % Check if A is SPD
        for j = 1 : n
            L(j,j) = (A(j,j) - sum(L(j,1:j-1).^2))^(1/2); 
            if(L(j,j) == 0 || ~isreal(L(j,j)))
                fprintf("The matrix is not positive definite")
                result = 0;
                return 
            end
            for i = j + 1 : n
                L(i,j) = (A(i,j) - sum(L(i,1:j-1).*L(j,1:j-1)))/L(j,j); 
            end
        end
        result = L;
    end
    \end{python}
    In the following code snippets, we preform the instruction from part (b):
    \begin{python}
    function question2()
        At = rand(10);  %generate random guy
    
        A = At + At';    
    
        L = cholesky(A);
        
        if(length(L) > 1)
            fprintf("The matrix is postive definte\n")
            mineval = min(eig(A));  %get smallest eigevalue
            if(mineval ~= 0 && isreal(mineval))
                fprintf("The minimal eigenvalue is %d which is positive and real\n",mineval)
            else
                fprintf("The algorithm failed")
            end 
            err = norm(L - chol(A,'lower'));
            fprintf("The norm error of the algorithm and matlab cholesky is %d\n",err)
        end
    end
    \end{python}
    which outputted \verb+The matrix is not positive definite+ over multiple attempts.

    \item [(c)]
    To preform part (c), I modified how we defined A as
    \begin{python}
    function question2()
        At = rand(10);  %generate random guy
    
        A = At'*At;    
    
        L = cholesky(A);
        
        if(length(L) > 1)
            fprintf("The matrix is postive definte\n")
            mineval = min(eig(A));  %get smallest eigevalue
            if(mineval ~= 0 && isreal(mineval))
                fprintf("The minimal eigenvalue is %d which is positive and real\n",mineval)
            else
                fprintf("The algorithm failed")
            end 
            err = norm(L - chol(A,'lower'));
            fprintf("The norm error of the algorithm and matlab cholesky is %d\n",err)
        end
    end
    \end{python}
    Now running the question 2 gives the following output:
    \begin{verbatim}
    The matrix is postive definte
    The minimal eigenvalue is 4.880531e-03 which is positive and real
    The norm error of the algorithm and matlab cholesky is 9.828340e-15
    \end{verbatim}


\end{enumerate}


Code can be found at \url{https://github.com/MarvynBailly/AMSC660/tree/main/homework5}.



\end{proof}
\end{problem}



%---------------%
%---Problem 3---%
%---------------%


\begin{problem}%[vskip]
\subsection*{Problem 3}

An $n \times n$ matrix is called \textit{tridiagonal} if it is of the form
\[
     A = \begin{pmatrix}
        b_1 & c_1 & 0 & \cdots & 0\\
        a_2 & b_2 & c_2\\
        0 & a_3 & b_3 & c_3\\
        && \ddots & \ddots & \ddots\\
        0 & \cdots & 0 & a_n & b_n 
     \end{pmatrix}
\]
There is a fast algorithm for solving linear systems of $Ay = f$ with invertible and strickly diagonal dominant (i.e. $|b_i| > |a_i| + |c_i| \forall i$) tridiagonal matrices $A$. Sometimes it is referred to as the \textit{Thomas algorithm:} 
\begin{verbatim}
function TridiagSolver(a,b,c,f)
    n = length(f);
    v = zeros(n,1);
    y = v;
    w = b(1);
    y(1) = f(1)/w;
    for i=2:n
        v(i-1) = c(i-1)/w;
        w = b(i) - a(i)*v(i-1);
        y(i) = ( f(i) - a(i)*y(i-1) )/w;
    end
    for j=n-1:-1:1
        y(j) = y(j) - v(j)*y(j+1);
    end
end
\end{verbatim}
Calculate the number of flops for the Thomas algorithm. 

\subsection*{Solution}
\begin{proof}

We wish to count the number of flops in the Thomas algorithm. Observe that
\begin{verbatim}
    function TridiagSolver(a,b,c,f)
        n = length(f);
        v = zeros(n,1);
        y = v;
        w = b(1);
        y(1) = f(1)/w;                      %one flop: "/"
        for i=2:n
            v(i-1) = c(i-1)/w;              % one flop: "/" 
            w = b(i) - a(i)*v(i-1);         % two flops: "-" and "*"
            y(i) = ( f(i) - a(i)*y(i-1) )/w;% three flops: "-", "*", and "/"
        end
        for j=n-1:-1:1
            y(j) = y(j) - v(j)*y(j+1);      % two flops: "-" and "*"
        end
    end
\end{verbatim}
Collecting the flops yields
\[
     W(n) = 1 + \sum_{i=2}^n 6 + \sum_{i=1}^{n-1}2 = 1 + 6(n-1) + 2(n-1) = 8n  - 7
\]

\end{proof}
\end{problem}




%---------------%
%---Problem 4---%
%---------------%


\begin{problem}%[vskip]
\subsection*{Problem 4}

Calculate the number of flops for the modified Gram-Schmidt algorithm for computing the QR factorization of an $n \times n$ matrix $A$. Here is a vectorized Matlab code implementing the modified Gram-Schmidt.
\begin{verbatim}
A = rand(n);
Q = zeros(n); R = zeros(n);
for i = 1 : n
    Q(:,i) = A(:,i);
    for j = 1 : i-1
        R(j,i) = Q(:,j)'*Q(:,i);
        Q(:,i) = Q(:,i) - R(j,i)*Q(:,j);
    end
    R(i,i) = norm(Q(:,i));
    Q(:,i) = Q(:,i)/R(i,i);
end

\end{verbatim}
\textit{Hint: The command Q(:,j)'*Q(:,i) means $\sum_{k=1}^nQ_{kj}Q{ki}$ and the command Q(:,i) = Q(:,i) - R(j,i)*Q(k,j) means the for-loop}
\begin{verbatim}
for k = 1 : n
    Q(k,i) = Q(k,i) - R(j,i)*Q(k,j);
end
\end{verbatim}

\subsection*{Solution}
\begin{proof}

We wish to calculate the number of flops in the following modified Gram-Schmidt algorithm
\begin{verbatim}
A = rand(n);
Q = zeros(n); 
R = zeros(n);
for i = 1 : n
    Q(:,i) = A(:,i);
    for j = 1 : i-1
        R(j,i) = Q(:,j)'*Q(:,i);    % 2n-1 flops: n "*" and n-1 "+"
        Q(:,i) = Q(:,i) - R(j,i)*Q(:,j);    % 2n flops: n "*" and n "-"
    end
    R(i,i) = norm(Q(:,i));      % 2n flops: n "*", n-1 "+", and "sqrt"
    Q(:,i) = Q(:,i)/R(i,i);                 % n flop: "/"
end
\end{verbatim}

Collecting the flops gives
\begin{align*}
    W(n) &= \sum_{i=1}^n\paren{\sum_{j=1}^{i-1}\paren{2n-1 + 2n} + 2n + n}\\
    &= \sum_{i=1}^n (2n-1)(i-1) + 2n(i-1) + 3n\\
    &\approx \int_0^n (2n-1)(x-1) + 2n(x - 1) + 3n \d x\\
    &= 2n^3 + \O(n^2).
\end{align*}
Thus the amount of flops the modified Gram-Schmidt algorithm takes is approximately $2n^3$.



\end{proof}
\end{problem}




%---------------%
%---Problem 5---%
%---------------%

\def\trace{{\text{trace}}}

\begin{problem}%[vskip]
\subsection*{Problem 5}

\begin{enumerate}
    \item [(a)] Prove the cyclic property of the trace:
    \[
         \trace(ABC) = \trace(BCA) = \trace(CAB)
    \]
    for all $A,B,C$ such that their product is defined and is a square matrix.
    \item [(b)] Prove that
    \[
        \|A\|_F^2 = \sum_{i=1}^d \sigma_i^2.
    \]
    \textit{Hint: use the full SVD of $A$ and the cyclic property of trace.}
    \item [(c)] Prove that
    \[
         \|A + B \|_F^2 = \|A\|_F^2 + \|B\|_F^2 + 2 \abrac{A,B}_F,
    \]
    where $\abrac{A,B}_F$ is the Frobenius inner product. The Frobenius inner product is defined as
    \[
         \abrac{A,B}_F := \sum_{i,j}a_{ij}b_{ij} = \trace(A^TB) = \trace(B^TA)
    \]
    


\end{enumerate}

\subsection*{Solution}
\begin{proof}

\begin{enumerate}
    \item [(a)]
    Consider the matrices $A$ be $n \times a$, $B$ be $a \times b$, and $C$ be $b \times n$. Let $D = AB$ which is a $n \times b$ matrix. Then observe that 
    \begin{align*}
        \trace(DC) &= \sum_{i} (DC)_{ii}\\
                &= \sum_{i}\sum_{j} d_{ij}c_{ji}\\    
                &= \sum_{i}\sum_{j} c_{ij}d_{ji}\\    
                &= \sum_{i} (CD)_{ii}\\
                &= \trace(CD),    
    \end{align*}
    thus we have that $\trace(ABC) = \trace(DC) = \trace(CD) = \trace(CAB)$. Now if we let $E = BC$ and apply what we found above, we get $\trace(BCA) = \trace(EA) = \trace(AE) = \trace(ABC)$. Thus we have that
    \[
        \trace(CAB) = \trace(ABC) = \trace(BCA).
    \]
    
    \item [(b)]
    Let $A = U \Sigma V^T$ be the SVD decomposition of $A$. Recall that
    \[
         \|A\|_F^2 = \trace(AA^T).
    \]
    Thus we have that
    \begin{align*}
        \|A\|_F^2 &= \trace\paren{AA^T}\\
                &= \trace\paren{U\Sigma V^T V \Sigma U^T}\\
                &= \trace\paren{U\Sigma^2 U^T}\\
                &= \trace\paren{\Sigma^2 U^TU}\\
                &= \trace{\paren{\Sigma^2}}\\
                &= \sum_{i=1}^d\sigma_i^2
    \end{align*}

    \item[(c)] 
    Let $ A = U \Sigma V^T$ and $B = \tilde{U}\tilde{\Sigma}\tilde{V}^T$ be the SVD decomposition of $A$ and $B$ respectively. Observe that
    \begin{align*}
        \|A + B \|_F^2 &= \trace\paren{(A + B)(A + B)^T}\\
        &= \trace\paren{\paren{U \Sigma V^T + \tilde{U}\tilde{\Sigma}\tilde{V}^T}\paren{U \Sigma V^T + \tilde{U}\tilde{\Sigma}\tilde{V}^T}^T}\\
        &= \trace\paren{\paren{U \Sigma V^T + \tilde{U}\tilde{\Sigma}\tilde{V}^T}\paren{\tilde{V}\tilde{\Sigma}\tilde{U}^T + V \Sigma U^T}}\\
        &= \trace\paren{U \Sigma V^T\tilde{V}\tilde{\Sigma}\tilde{U}^T + \tilde{U}\tilde{\Sigma}\tilde{V}^T\tilde{V}\tilde{\Sigma}\tilde{U}^T + U \Sigma V^TV \Sigma U^T + \tilde{U}\tilde{\Sigma}\tilde{V}^TV \Sigma U^T}\\
        &= \trace\paren{AB^T + \tilde{U}\tilde{\Sigma}^2\tilde{U}^T + U \Sigma^2U^T + BA^T}.
    \end{align*}
    As the $\trace$ is a linear mapping and $\trace(AB^T) = \trace(B^TA) = \trace(BA^T)$ we get
    \begin{align*}
        \|A + B \|_F^2 &= 2 \trace\paren{AB^T} + \trace\paren{\tilde{U}\tilde{\Sigma}^2\tilde{U}^T} + \trace\paren{U \Sigma^2U^T}\\
        &= 2 \trace\paren{AB^T} + \trace\paren{\tilde{\Sigma}^2} + \trace\paren{\Sigma^2}\\
        &= \|A\|_F^2 + \|B\|_F^2 + 2 \abrac{A,B}_F. 
    \end{align*}
    Therefore $\|A + B \|_F^2 = \|A\|_F^2 + \|B\|_F^2 + 2 \abrac{A,B}_F$


\end{enumerate}

\end{proof}
\end{problem}








\end{document}